%%%%%%%%%%%%%%%%%%%%%%%%%%%%%%%%%%%%%%%%%
%%                                     %%
%%     LaTeX template for CCS 2019     %%
%%        abstract submissions         %%
%%                                     %%
%%          <30 March 2019>            %%
%%                                     %%
%% Author: Liu Wenyuan                 %%
%% Eamil: liuw0037@ntu.edu.sg          %%
%% Please contact author if you have   %%
%% any problem using this template     %%
%%%%%%%%%%%%%%%%%%%%%%%%%%%%%%%%%%%%%%%%%

\documentclass[11pt, a4paper]{article}
\usepackage{amsmath}
\usepackage{amsfonts}
\usepackage{amssymb}
\usepackage{graphicx}
\usepackage{hyperref}
\usepackage[top=3.17cm, bottom=2.54cm, left=2.54cm, right=2.54cm, includehead, includefoot,]{geometry}
\usepackage{authblk}
\usepackage{tikz}

\renewcommand{\refname}{References}
%Only use following two lines if you want to use Times New Roman font
%Only work with XeLaTex or LuaLaTex
%\usepackage{fontspec}
%\setmainfont{Times New Roman}

\newcommand{\va}{\boldsymbol{a}}
\newcommand{\vu}{\boldsymbol{u}}
\newcommand{\vE}{\boldsymbol{E}}

\begin{document}
\title{\vspace{-3.0cm}\large\textbf{`Financial Contagion': Application of the
    Structural Model of
    Credit Risk\\to the Network of Interbank Loans}}

\author[*]{\footnotesize A.\@ Tucker}
\affil[*]{\footnotesize University of Warwick, Coventry, UK, \url{agjf.tucker@gmail.com}}
\date{\vspace{-5ex}}

\maketitle
\thispagestyle{empty}

\noindent
By viewing the debt of an entity as an option on its assets, the structural
model of credit risk \cite{merton} provides a principled means by which to price risk of default.
Known as CVA (credit valuation adjustment), that price has very real
consequences.
The Basel
Committee on Banking Supervision has estimated that, during
the financial crisis,
two thirds of losses attributed to counterparty credit risk were due to
CVA losses and only one third to actual defaults.
Network models designed to take account of the interconnectedness of the
financial system, beginning with Eisenberg and Noe \cite{eisenberg-noe}, have traditionally
employed a fixed point argument to assess only disruption communicated through actual default.
Recently however the Bank of England has published details of its own
`solvency contagion model' \cite{bardoscia} attempting to marry the two
approaches.
Despite use of the word `contagion', the problem motivates a treatment quite
distinct from any found in epidemiology.
I will argue that the Bank of England's approach is na\"ive and that a return to the principles behind the Merton model
is necessary to rederive the equations in the correct generality.
The PDE becomes $n$-dimensional and its solution a
more demanding task.
Moreover it becomes clear
that the fixed point approach has served to conceal subtle issues.
But the computation is feasible and the issues may be resolved. Our reward is a solid foundation to this
important calculation.

\vspace{1\baselineskip}\noindent
State may be characterised
by set of surviving banks $s$, vector of asset prices
$\va\in \mathbb{R}_{+}^{n}$ and time $t\in[0,T)$ up to final time $T\in\mathbb{R}_{+}\cup\{\infty\}$.
My main result is that a system of domains
$V_{s}\subseteq\mathbb{R}_{+}^{n}\times [0,T)$ and \emph{continuous} debt valuation functions
$\vu_{s}:\left\{(\va,t,\tau)\,\vert\, 0\leq t<\tau\leq T\right\}\to [0,1]^{n}$
($\tau$ is maturity) can be constructed to satisfy, and is uniquely determined by,
these properties holding for each $s$:
\begin{itemize}
\item
  that with $\vE^{*}(\vu_{s},\va,t)$ the
  equity valuation function associated with $s$, we have
\(
  (\va,t)\in V_{s} \) if and only if \(\forall i \in s, 0<E^{*}_{i}(\vu_{s},\va,t)
\),
  \item
that for $i\in s$ the corresponding component of $\vu_{s}$
solve on \(V_{s}\) the multidimensional Black-Scholes-Merton PDE,
\begin{equation*}
  \frac{\partial u_{i}}{\partial t} +
  \frac{1}{2}\sum_{jk} S_{jk}a_{j}a_{k}\frac{\partial^{2} u_{i}}{\partial
    a_{j}\partial a_{k}}
  +r\sum_{j}a_{j}\frac{\partial{u_{i}}}{\partial{a_{j}}}-ru_{i}+c
  =0,
\end{equation*}
\item
  that elsewhere \( \vu_{s}(\va,t,\tau)=\vu_{s'}(\va,t,\tau) \) for $s'$ a maximal element
  of $\left\{r\subseteq s\,\vert\, (\va,t)\in
      V_{r}\right\}$,
\item
  that for $i\in s$ and $(\va,t)\in V_{s}$ the corresponding component of $\vu_{s}$ satisfy
\(
  \lim_{\tau\to t^{+}}u_{i}(\va,t,\tau)=1
\), and
\item
  that for $i\notin s$ the corresponding component of \(\vu_{s}\)
  be zero.
\end{itemize}
I have performed calculations for up to seven banks (sufficient for the
UK system) using my own Python/SciPy code.
While the Bank of England has made public its algorithm, its data are not, so
the only numerical comparisons I can make are using fictitious
figures.

\begin{figure}[!t]
  \centering
  \input{discontinuity_0.pgf}
  \input{discontinuity_1.pgf}
  \caption{The Bank of England's algorithm can throw up some unwanted artefacts. Here
    we see its valuation of BARC debt (left) and RBSG debt (right) where both survive
    (blue) and where one has failed (red). On the left, note the
    discontinuity in the value of BARC debt at the boundary where RBSG fails. On
    the right, the discontinuity in the value of RBSG debt at
    the boundary where BARC fails.
    This is the consequence of multiple solutions and the decision to
    select the most optimistic of them.}
  \label{fig:discontinuities}
\end{figure}

\begin{figure}[!htp]
  \centering
  \input{continuity_0.pgf}
  \input{continuity_1.pgf}
  \caption{There is no possibility of multiple solutions in the model I
    propose, and there are no discontinuities in pricing at boundaries. This
    accords with intuition. The theory of random walks tells us that as we come close to a
    boundary the probability of crossing it approaches $1$, so our expectation
    should make no sudden jump as we do so.}
\end{figure}

\begin{thebibliography}{9}
\bibitem{merton}
Merton, R.\@ C.\@ (1974) On the pricing of corporate debt: The risk structure of interest rates. \emph{The Journal of Finance}, 29(2):449--470.
\bibitem{eisenberg-noe}
Eisenberg, L., \& Noe, T.\@ H.\@ (2001) Systemic risk in financial systems. \emph{Management
Science}, 47(2):236--249.
\bibitem{bardoscia}
Bardoscia, M., Barucca, P., Brinley Codd, A., \& Hill, J.\@ (2017) The decline of solvency contagion risk. Bank of England working papers 662, Bank of England.
\end{thebibliography}

\end{document} 